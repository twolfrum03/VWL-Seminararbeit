% ------------------------------------------------------------

% LaTeX Template für die DHBW zum Schnellstart!
% Original: https://github.wdf.sap.corp/vtgermany/LaTeX-Template-DHBW
% ------------------------------------------------------------
% ---- Präambel mit Angaben zum Dokument
\input{Inhalt/00_Latex/praeambel}

% ---- Elektronische Version oder Gedruckte Version?
% ---- Unterschied: Die elektronische Version enthält keinen Platzhalter für die Unterschrift
\usepackage{ifthen}
\usepackage{color}
\newboolean{e-Abgabe}
\setboolean{e-Abgabe}{false}    % false=gedruckte Fassung

% ---- Persönlichen Daten:
\newcommand{\titel}{Die Kosten- und Erlösrechnung als Instrument der Planung und Kontrolle in Industriebetrieben}
\newcommand{\titelheader}{Kosten- und Erlösrechnung in Industriebetrieben}
\newcommand{\arbeit}{wissenschaftliche Ausarbeitung}
\newcommand{\studiengang}{Wirtschaftsinformatik}
\newcommand{\studienjahr}{2023}
\newcommand{\autor}{Tom Wolfrum}
\newcommand{\autorReverse}{Wolfrum, Tom}
\newcommand{\verfassungsort}{Karlsruhe}
\newcommand{\matrikelnr}{4000776}
\newcommand{\kurs}{WWI22B5}
% \newcommand{\bearbeitungsmonat}{Januar 2018}
\newcommand{\abgabe}{12. November 2023}
\newcommand{\bearbeitungszeitraum}{16.10.2023 - 12.11.2023}
\newcommand{\firmaName}{SAP SE}
\newcommand{\firmaStrasse}{Dietmar-Hopp-Allee 16}
\newcommand{\firmaPlz}{69190 Walldorf, Deutschland}
\newcommand{\betreuerFirma}{Steven Rösinger}
\newcommand{\betreuerDhbw}{Harald Haake}

\input{Inhalt/00_Latex/kopfundFusszeile}

% ---- Hilfreiches
\newcommand{\zB}{z.\,B. }   % "z.B." mit kleinem Leeraum dazwischen (ohne wäre nicht korrekt)
\newcommand{\dash}{d.\,h. }

\newcommand{\code}[1]{\texttt{#1}} % Ist einfacher zu schreiben als ständig \texttt und erlaubt
                                   % Änderungen im Nachhinein, wenn man z.B. Inline-Code anders stylen möchte.

% ---- Silbentrennung (falls LaTeX defaults falsch / nicht gewünscht sind)
\hyphenation{HANA}         % anstatt HA-NA
\hyphenation{Graph-Script} % anstatt GraphS-cript

% ---- Beginn des Dokuments

\begin{document}
\setlength{\parindent}{0pt}              % Keine Paragraphen Einrückung.
                                         % Dafür haben wir den Abstand zwischen den Paragraphen.
\setcounter{secnumdepth}{2}              % Nummerierungstiefe fürs Inhaltsverzeichnis
\setcounter{tocdepth}{2}                 % Tiefe des Inhaltsverzeichnisses. Ggf. so anpassen,
                                         % dass das Verzeichnis auf eine Seite passt.
\sffamily                                % Serifenlose Schrift verwenden.

% ------ Vorspann
% ------ Titelseite
\singlespacing
\thispagestyle{empty}
\begin{titlepage}
\enlargethispage{4cm}

\begin{figure}           % Logo vom Ausbildungsbetrieb und der DHBW
	% \vspace*{-5mm} % Sollte dein Titel zu lang werden, kannst du mit diesem "Hack" 
	%                  den Inhalt der Seite nach oben schieben.
	\begin{minipage}{0.49\textwidth}
		\flushleft
		\includegraphics[height=2.5cm]{Bilder/Logos/Logo_SAP.pdf} 
	\end{minipage}
	\hfill
	\begin{minipage}{0.49\textwidth}
		\flushright
		\includegraphics[height=2.5cm]{Bilder/Logos/Logo_DHBW.pdf} 
	\end{minipage}
\end{figure} 
\vspace*{2.5cm}

\begin{center}
	\huge{\textbf{\titel}}\\[1.5cm]
	\Large{\textbf{\arbeit}}\\[0.5cm]
	\normalsize{im Rahmen der Portfolioprüfung in der Vorlesung\\[1ex] \textbf{Geld, Währung und Wirtschaftspolitik}}\\[0.5cm]
	\Large{des Studienganges \studiengang}\\[1ex]
	\normalsize{an der Dualen Hochschule Baden-Württemberg Karlsruhe}\\[1cm]
	\normalsize{von}\\[1ex] \Large{\textbf{\autor}} \\[1cm]

	% Hinweis: Manche Dozenten möchten einen Hinweis auf den Sperrvermerk auf der Titelseite.

	% Sperrvermerkt ein-/auskommentieren:
	% \large{{\color{red}- Sperrvermerk -}}\\[1cm]


\end{center}

\begin{center}
	\vfill
	\begin{tabular}{ll}
		\textbf{Abgabedatum:}                     & \abgabe \\[0.2cm]
		\textbf{Bearbeitungszeitraum:}            & \bearbeitungszeitraum \\[0.2cm]
		\textbf{Kurs:}            				  & \kurs \\[0.2cm]
		\textbf{Ausbildungsfirma:}                & \firmaName \\
		                                 		  & \firmaStrasse \\
		                                          & \firmaPlz \\[0.2cm]
		\textbf{Betreuer der Ausbildungsfirma:}   & \betreuerFirma \\[0.2cm]
		\textbf{Gutachter der Dualen Hochschule:} & \betreuerDhbw \\[2cm]
	\end{tabular} 
\end{center}
\end{titlepage}
  % Titelseite
\newcounter{savepage}
\pagenumbering{Roman}                    % Römische Seitenzahlen
\onehalfspacing

% ------ Erklärung, Sperrvermerk, Abstact
% \chapter*{Sperrvermerk}
Die nachfolgende Arbeit enthält vertrauliche Daten der:
\begin{quote}
	\firmaName \\
	\firmaStrasse \\
	\firmaPlz
\end{quote}

\vspace{0.5cm}

Der Inhalt dieser Arbeit darf weder als Ganzes noch in Auszügen Personen ausserhalb des Prüfungs- und Evaluationsverfahrens zugänglich gemacht werden, sofern keine anders lautende Genehmigung des Dualen Partners vorliegt.
\chapter*{Selbstständigkeitserklärung}

Ich versichere hiermit, dass ich die vorliegende \arbeit{} mit dem Thema:
\begin{quote}
	\textit{\titel}
\end{quote}
selbstständig verfasst und keine anderen als die angegebenen Quellen und Hilfsmittel benutzt habe.

\vspace{0.25cm}


\vspace{1cm}

\verfassungsort, den \today \\[0.5cm]
\ifthenelse{\boolean{e-Abgabe}}
	{\underline{Gez. \autor}}
	{\makebox[6cm]{\hrulefill}}\\ 
\autorReverse

\chapter*{Geschlechtsneutrale Formulierung}

 

In dieser Arbeit wird aus Gründen der besseren Lesbarkeit das generische Maskulinum verwendet.

Weibliche und anderweitige Geschlechteridentitäten werden dabei ausdrücklich mitgemeint, soweit es für die Aussage erforderlich ist.

%\include{Inhalt/02_Abstract/abstract-en}y
%\include{Inhalt/02_Abstract/abstract-de}

% ------ Inhaltsverzeichnis
\singlespacing
\small
\tableofcontents
\normalsize

% ------ Verzeichnisse
\renewcommand*{\chapterpagestyle}{plain}
\pagestyle{plain}
%\include{Inhalt/03_Verzeichnisse/formelgroessen}
\chapter*{Abkürzungsverzeichnis}
\addcontentsline{toc}{chapter}{Abkürzungsverzeichnis} % Hinzufügen zum Inhaltsverzeichnis 

\begin{acronym}[WYSISWG] % längstes Kürzel wird verw. für den Abstand zw. Kürzel u. Text

	% Alphabetisch selbst sortieren - nicht verwendete Kürzel rausnehmen!
	
	% Bsp.:
	\acro{T10/B50}{gewichtetes Verhältnis des Durchschnittseinkommens der reichsten 10\% zum Durchschnittseinkommen der ärmsten 50\%}

\end{acronym}
\listoffigures                          % Erzeugen des Abbildungsverzeichnisses 
\listoftables                           % Erzeugen des Tabellenverzeichnisses
\renewcommand{\lstlistlistingname}{Quellcodeverzeichnis}
%\lstlistoflistings                      % Erzeugen des Listenverzeichnisses
\setcounter{savepage}{\value{page}}


% ------ Inhalt der Arbeit
\cleardoublepage
\pagenumbering{arabic}                  % Arabische Seitenzahlen für den Hauptteil
\setlength{\parskip}{0.5\baselineskip}  % Abstand zwischen Absätzen
\rmfamily
\renewcommand*{\chapterpagestyle}{scrheadings}
\pagestyle{scrheadings}
\onehalfspacing
%\include{Inhalt/04_Inhalt/einleitung}
%\include{Inhalt/04_Inhalt/formatText}
%\include{Inhalt/04_Inhalt/abbildungen}
%\include{Inhalt/04_Inhalt/mathematische-formeln}
%\include{Inhalt/04_Inhalt/quellcode}
%\include{Inhalt/04_Inhalt/literaturHinweis}

% \include{Inhalt/04_Inhalt/einleitung.tex}
% \include{Inhalt/04_Inhalt/grundlagen.tex}
% \include{Inhalt/04_Inhalt/dynamischeLearningNFTs.tex}
% \include{Inhalt/04_Inhalt/anwendbarkeitfürXGP.tex}
% \include{Inhalt/04_Inhalt/schlussbetrachtung.tex}

\chapter{Einleitung}

Die Planung und Kontrolle von Produktions- und Absatzaktivitäten ist ein zentrales Element moderner Betriebswirtschaft und Gegenstand zahlreicher theoretischer Ansätze. Einer dieser Ansätze ist das Modell der Betriebsplankosten- und Betriebsplanerfolgsrechnung von Gert La{\ss}mann von 1968. Dieses Modell bietet einen detaillierten und gründlichen Rahmen für die Berücksichtigung aller zentralen Einflussfaktoren auf den Periodenerfolg und hebt sich durch seine Akzentuierung von intensiver Planung und Überwachung ab, insbesondere in sich Branchen mit sich schnell verändernden Rahmenbedingungen, wie der Montantindustrie. Vor diesem Hintergrund zielt diese Arbeit darauf ab, die theoretischen Konzepte von Gert La{\ss}mann umfassend vorzustellen und sie auf ein konkretes Praxisbeispiel anzuwenden. Hierfür wird die fiktive HobelKunstwerkstatt GmbH als Fallstudie herangezogen. Die Arbeit verfolgt das Ziel, durch die theoretische Betrachtung und praktische Anwendung der Betriebsplankosten- und Betriebsplanerfolgsrechnung detaillierte Einblicke zu gewinnen und eine fundierte Bewertung des Modells im Kontext realer Geschäftsszenarien zu ermöglichen. Abschlie{\ss}end wird eine Bewertung des Ansatzes vorgenommen, die seine Stärken und möglichen Schwachstellen dezidiert betrachtet.

\chapter{Theoretische Grundlagen}

\section{Grundlagen der Betriebsplankosten- und Betriebsplanerfolgsrechnung}

\subsection{Betriebs- und Absatzmodelle}

Die Betriebsplankosten- und -erfolgsrechnung ist eine Weiterentwicklung der flexiblen Plankostenrechnung und Deckungsbeitragsrechnung. Sie verknüpft das betriebliche Rechnungswesen direkt mit den Planungs- und Überwachungsprozessen in Absatz und Produktion. Alle wichtigen Einflussgrö{\ss}en auf den Periodenerfolg müssen in Planung und Abrechnung erfasst werden. Als materielle Basis dienen Betriebs- und Absatzmodelle, die Einflussgrö{\ss}enbeziehungen in mathematischen Funktionen abbilden. Nur solche Einflussgrö{\ss}enbeziehungen werden berücksichtigt, die für die Planung und Überwachung von Produktion und Absatz sowie für die Produktkalkulation und Preisbeurteilung entscheidend sind.

Betriebsmodelle basieren auf engineering production functions oder Prozessmodellen, die zur Steuerung technologischer Produktionsabläufe eingesetzt werden. Beispiele dafür sind Hochofenprozessmodell, Stahlwerksprozessmodell oder verfahrenstechnische Prozessmodelle in der Chemieindustrie. Diese Prozessmodelle sind oft zu feingliedrig und komplex für betriebswirtschaftliche Aufgaben, insbesondere hinsichtlich der Wirtschaftlichkeit des Rechnungswesens. Daher müssen aus ihnen die betriebswirtschaftlich relevanten Einflussgrö{\ss}enbeziehungen abgeleitet werden. Einige Modelle beinhalten nichtlineare Einflussgrö{\ss}enfunktionen, bei denen oft eine (abschnittsweise) lineare Approximation für betriebswirtschaftliche Planungsansätze ausreicht. In einer produktionstheoretischen Sicht basieren Betriebsmodelle auf den Input-Output-Modellen von Leontief (1953), Betriebsmatrizen von Pichler (1961) und Verbrauchsfunktionen von Gutenberg (1983).

Absatzmodelle basieren in der Regel auf der Analyse vergangener Absatzprozesse und Projektionen zukünftiger Absatzaktivitäten. Absatzleistungsarten und entsprechend bewerteter Erlösarten transparent machen. Auf diese Basis können marktsegmentspezifische Absatzmodelle erstellt werden. Absatzmodelle stellen die verschiedenartigen Absatzleistungen in Abhängigkeit von ihren Haupteinflussgrö{\ss}en dar und unterstützen Prognosen über zukünftige Absatzentwicklungen. Für die Ableitung von Absatz- und Erlösplänen sowie spezifischen Vorgaben im Vertriebsbereich müssen zusätzlich die Einflüsse von geplanten Absatzaktivitäten und bereits vorhandenen Auftragsbeständen berücksichtigt werden. Die auf den Absatzmodellen basierende Planerlösrechnung hat mindestens ebenso hohe Bedeutung wie die Betriebsplankostenrechnung. Ohne Kenntnis der wichtigsten Erlöseinflussgrö{\ss}en und der im Absatz vorherrschenden Wirkungszusammenhänge ist eine fundierte Absatz- und Erlösplanung nicht möglich.

\subsection{Periodenerfolg als Lenkungsziel}

Wirtschaftliches Ziel von Unternehmen ist die Erreichung hoher Periodenerfolge. Die Betriebsplankosten- und -erfolgsrechnung basiert auf Betriebs- und Absatzmodellen und zielt auf die Ermittlung erfolgsoptimaler Produktions- und Absatzpläne ab. Die Betriebsplankosten- und -erfolgsrechnung dient zur Überwachung der Planumsetzung, differenziert nach den wichtigsten Erfolgskomponenten. Auf Basis von Einflussgrö{\ss}enfunktionen werden für alternative Produkt- und Absatzprogramme, verfügbare Produktions- und Absatzbedingungen und/oder Faktoreinsatzzusammensetzungen die zu erwartenden Periodenkosten und -erlöse ermittelt. Der Produktions- und Absatzplan kann durch lineare Programmierung optimiert werden, sofern ausreichende Informationen vorhanden sind. Im Gegensatz zur flexiblen Plankosten- und Deckungsbeitragsrechnung dient die Betriebsplankosten- und -erfolgsrechnung gleichzeitig der periodenbezogenen Produktions- und Absatzplanung sowie der Planung und Überwachung der periodenbezogenen Produktionskosten und Absatzerlöse. Mit der Betriebsplankosten- und -erfolgsrechnung können auch Stückkosten und -erlöse je Produktions- und Absatzbereich ermittelt werden. Die Ursachen für Plan-Ist-Abweichungen können detaillierter ermittelt werden als in der flexiblen Plankostenrechnung. Der Ansatz kann auch alle notwendigen Dokumentationsanforderungen erfüllen.

\section{Aufbau der Betriebsplankosten- und Betriebsplanerfolgsrechnung}

\subsection{Ermittlung von Einflu{\ss}grö{\ss}enfunktionen}

Der Aufbau einer Betriebsplankosten- und -erfolgsrechnung beginnt mit der Bestimmung wesentlicher Kosten- und Erlöseinflussgrö{\ss}en, die als Ursachen von Kostenverbräuchen oder betrieblichen Faktoreinsätzen angesehen werden können. Die Disponibilität der Einflussgrö{\ss}en ist ein wichtiges Kriterium, dabei gibt es Einflussgrö{\ss}en, die frei von der Betriebsleitung verfügbar sind und solche, die extern bestimmt sind. Neben primären Einflussgrö{\ss}en, die von der Betriebsleitung oder der "Umwelt" bestimmt werden, gibt es sekundäre Einflussgrö{\ss}en, die als Zwischenergebnisse in der betrieblichen Einflussgrö{\ss}enrechnung anfallen. Methoden zur Bestimmung der Wirkungsweise von Einflussgrö{\ss}en auf Faktoreinsätze und Absatzleistungen umfassen statistische und analytische Verfahren. Bei statistischen Verfahren werden aus vergangenen Istwerten Faktoreinsatz- und Absatzleistungsfunktionen abgeleitet, bei analytischen Verfahren werden diese Beziehungen auf der Grundlage theoretischer Studien und wissenschaftlicher Erkenntnisse festgelegt. Regelmä{\ss}ige Überprüfungen der Einflussgrö{\ss}enfunktionen sind notwendig, da der Gültigkeitsbereich der statistisch ermittelten Koeffizienten begrenzt ist und Veränderungen im Zeitverlauf eine Neubestimmung der Koeffizienten erfordern können. Ein Beispiel für eine konkrete Betriebsstoffeinsatzfunktion zeigt die Grundstruktur von Einflussgrö{\ss}enfunktionen eines Betriebsmodells.

\subsection{Strukturelemente von Betriebs- und Absatzmodellen}

Im ersten Schritt zur Erstellung einer Betriebsplankosten- und -erfolgsrechnung erfolgt die Bestimmung wesentlicher Kosten- und Erlöseinflussgrö{\ss}en. Einflussgrö{\ss}en, die als Ursachen für Kostengüterverbräuche, betriebliche Faktoreinsätze und Absatzleistungen betrachtet werden, bilden unabhängige Variablen. Einflussgrö{\ss}en können von Betriebsleitung frei verfügbar sein oder durch externe Faktoren bestimmt werden. Neben den primären Einflussgrö{\ss}en gibt es sekundäre Einflussgrö{\ss}en, die als Zwischenergebnisse in der betrieblichen Einflussgrö{\ss}enrechnung anfallen. Die Wirkungsweise von Einflussgrö{\ss}en auf ausgewählte Faktoreinsätze und Absatzleistungen kann durch statistische und analytische Verfahren bestimmt werden. Die Strukturmatrix stellt ein allgemeines Ordnungsschema für Vektoren und Matrizen dar, mit dem Produktions- und Absatzprozesse abgebildet werden. Die Strukturmatrix beinhaltet Vorgabegrö{\ss}en oder primäre Einflussgrö{\ss}en sowie abgeleitete Zwischenzielgrö{\ss}en und resultierende Kostengüterverbrauchs-Zielgrö{\ss}en. In den Feldern der Strukturmatrix sind Koeffizientenmatrizen untergebracht. Die letzte Zeile der Strukturmatrix enthält technologische oder durch Umwelt und Betriebsleitung bestimmte Einschränkungen. Ähnlich aufgebaut ist die Strukturmatrix eines Absatzmodells, wo im ersten Schritt das Absatzmengenprogramm, differenziert nach Qualitäten und/oder Abmessungen, bestimmt wird. Aufgrund der Unberechenbarkeit des Marktverhaltens ist meist das differenzierte Absatzprogramm selbst eine Vorgabe. Für die Monats- und Quartalsplanung von Absatz und Produktion bzw. Kosten und Erlösen können oft Vorgaben für den Absatzbereich aus vorhandenen Auftragsbeständen abgeleitet werden. Abschlie{\ss}end wird das Absatzprogramm, differenziert nach Vertriebswegen, sonstigen Dienstleistungen und anderen erlösbestimmenden Komponenten ermittelt.

\subsection{Verknüpfung von Betriebs- und Absatzmodellen zum Periodenerfolgsmodell}

Betriebs- und Absatzmodelle stellen Mengenbeziehungen zwischen Produktmengen und deren Haupteinflussgrö{\ss}en dar und haben den Charakter von Input-Output-Modellen. Durch multiplikative Verknüpfung mit Preisvektoren entstehen Planerlöse und Plankosten je Periode, differenziert nach Absatzleistungsarten und Kostenarten. Der Planperiodenerfolg wird als Differenz zwischen den Gesamtperiodenerlösen und -kosten berechnet. Bei der Verknüpfung aufeinanderfolgender Periodenmodelle sind Bestandszuführungen und -entnahmen von Halb- und Fertigfabrikaten zu beachten. Es können sowohl Alternativkalkulationen durchgeführt als auch Optimierungsrechnungen unter Einsatz linearer Programmierungsansätze angewendet werden.

\section{Einsatz der Betriebsplankosten- und -erfolgsrechnung im Unternehmenscontrolling bei Sorten- und Serienfertigung}

\subsection{Integrierte Produktions-, Absatz-, Kosten-, Erlös- und Erfolgsplanung}

Betriebs- und Absatzmodelle reflektieren Mengenbeziehungen zwischen Produktionsprogramm und Haupteinflussgrö{\ss}en, visualisiert durch Input-Output-Modelle. Die multiplikative Verknüpfung mit Preisvektoren bildet Planerlöse und Plankosten pro Periode, differenziert nach Absatzleistungs- und Kostenarten. Wechsel im Produktionsprogramm und Umfeld erfordern Anpassungen im Herstellungsprozess, beispielsweise Verfahrenswechsel, In- oder Au{\ss}erbetriebnahme von Anlagen, Variation von Fertigungslosgrö{\ss}en etc. Wirtschaftliche Kriterien für solche Entscheidungen sind die Erfolgswirkungen verschiedener Ma{\ss}nahmen. Zwei separate Schritte sind erforderlich: eine Mengenberechnung und eine Bewertungsrechnung. Mengenberechnung berechnet Erzeugniseinsatzstoffmengen und Planzeiten der Systeme aus dem Produktionsprogramm und sonstigen Einflussgrö{\ss}en. Bewertungsrechnung berechnet die Periodenkosten, indem sie Kostengüterbedarfsmengen mit entsprechenden Preiskategorien verknüpft. Es besteht eine zunehmende Differenz zwischen Plankosten und aktuellem Preisniveau im Jahresverlauf. Vor allem bei kurz- und mittelfristigen Anpassungen an Beschäftigungsschwankungen kommen Optimierungsansätze zum Einsatz. Alternativen im Produktionsvollzug und alternative Absatzmengen erfordern wirtschaftliche Beurteilung. Die Bedingungen für einen optimalen Erfolg können unter bestimmten Voraussetzungen bestimmt werden. Simulationen können die Einflüsse verschiedener Rabattmengen, Rabattsatzgestaltungen, Bonusgewährungen, und Vertriebswegealternativen auf Periodenerlös evaluieren. Periodenerfolgsmaximale Produktions- und Absatzprogramme können ermittelt werden, wenn sowohl Kosten- als auch Erlöseinflussgrö{\ss}en verfügbar sind.

\subsection{Produktions-, Absatz-, Kosten-, Erlös-, und Erfolgsüberwachung}

Die Abweichungsanalyse ist ein wesentlicher Bestandteil der kurzfristigen Erfolgsermittlung und -überwachung und kann auf jedem gewünschten Detailgrad durchgeführt werden. Die Aussagekraft der Analyse hängt von der gewählten Zielsetzung und dem Differenzierungsgrad der Betriebs- und Absatzmodelle ab. Die Analyse kann Ursachen und Verantwortlichkeiten für Abweichungen aufzeigen, beispielsweise in Bereichen wie Leistung, Produktion, Material, Arbeitszeit etc. Die Abweichungsanalyse gliedert die Differenz zwischen Plan- und Isterfolg nach Erlös- und Kostenabweichungen und unterscheidet jeweils nach Preis- und Mengenabweichungen. Planänderungen und Verbrauchsabweichungen werden in der Kostengütermengenüberwachung detailliert ermittelt und unterschieden. Es sind Anpassungen an veränderte Umstände erforderlich, die sich in veränderten Kostengütermengenabweichungen niederschlagen. Abweichungen von den Sollgrö{\ss}en sind zu berücksichtigen, wenn sie au{\ss}erhalb der statischen Streubreite liegen. Solche Abweichungen können entweder negativ (z.B. durch Materialfehler oder Fehlverhalten der Arbeitskräfte) oder positiv (z.B. durch technische Verbesserungen oder höhere Arbeitseffizienz) sein. Die auf Entscheidungen beruhenden Abweichungen ergeben sich durch den Vergleich von am Monatsanfang berechneten Plankosten mit den Sollkosten, die nach Monatsende berechnet wurden. Am Monatsende können auf Basis der tatsächlichen Produktionsmengen und -bedingungen die Sollkosten berechnet und den Istkosten gegenübergestellt werden. Nur wesentliche Abweichungen, die vom Empfänger des Berichts verursacht und zu verantworten sind, sollten ermittelt und dokumentiert werden.

\subsection{Produktkalkulation zur Ermittlung von Preisgrenzen und Produkterfolgen}

Die Erzeugnis- und Auftragskosten und -erlöse werden insbesondere für die Bildung von Angebotspreisen, zur Beurteilung von Marktpreisen, zur Bewertung von Lagerbeständen und zur Ermittlung von Produkterfolgen benötigt. Mit der Betriebsplankosten- und -erfolgsrechnung kann die Kostenträgerrechnung in jeder gewünschten Kostenabgrenzung vollzogen werden. Um Kostenträgerrechnung durchzuführen, müssen bestimmte Vorgaben für alle Freiheitsgrade eingesetzt werden. Unter Risikogesichtspunkten können auch Alternativkalkulationen durchgeführt werden. Aufgrund der Trennung zwischen Mengen- und Bewertungsrechnung in der Endphase der Kostenermittlung können die Auswirkungen von Preisänderungen auf die Herstellungskosten gezeigt werden. In analoger Form erlauben Absatzmodelle die Durchführung einer Erlöskalkulation zur Bestimmung von Brutto- und Nettoerlösen je Einheit des Absatzprogramms. Durch Zusammenführung von Erlös- und Kostenkalkulation lässt sich eine Erfolgskalkulation pro Erzeugniseinheit oder Auftrag aufbauen. Stückerfolgsgrö{\ss}en haben für die Betriebsplankosten- und -erfolgsrechnung nur eine sekundäre Bedeutung, da sie für die periodenbezogene Produktions- und Absatzplanung nicht benötigt werden.

\section{Konzeptionelle Unterschiede zwischen Betriebsplankosten- und -erfolgsrechnung und flexibler Plankostenrechnung}

Die flexible Plankosten- und Deckungsbeitragsrechnung geht davon aus, dass sich die variablen Kosten und Erlöse allein zu Produktmengenvariationen je Periode proportional verhalten. Alle anderen Kosten- und Erlöseinflussgrö{\ss}en werden in der flexiblen Plankosten- und Deckungsbeitragsrechnung im Zuge vorgelagerter Planungsprozesse festgelegt. Die Betriebsplankosten- und -erfolgsrechnung hingegen erfasst alle wesentlichen Kosten- und Erlösabhängigkeiten eines Unternehmens und ermöglicht die Berechnung des Periodenerfolgs von Planungsalternativen. In der flexiblen Plankostenrechnung ist die Ermittlung einer Vielzahl von alternativ-konstanten Grenzplankosten für jede zu erwartende Kombination der Einflussgrö{\ss}enwerte notwendig. Die flexible Plankosten- und Deckungsbeitragsrechnung eignet sich insbesondere für die Kalkulation, Planung und Kontrolle von Produktions- und Absatzprozessen, während die Betriebsplankosten- und -erfolgsrechnung die periodenbezogene Kosten- und Erlös- bzw. Erfolgsermittlung im Zusammenhang mit der kurz- bis mittelfristigen Planung und Kontrolle von Produktion und Absatz ermöglicht. Bei der Betriebsplankosten- und -erfolgsrechnung erfolgt eine strikte Trennung von flexibler Mengen- und Bewertungsrechnung mit laufender Preisaktualisierung, während in der flexiblen Plankostenrechnung die Mengen- und Bewertungsrechnung integriert sind. In der Betriebsplankosten- und -erfolgsrechnung können Abweichungen nach technischen Einzelursachen untergliedert und unter Berücksichtigung von Verantwortlichkeiten analysiert werden, während in der flexiblen Plankostenrechnung Beschäftigungs- bzw. Bezugsgrö{\ss}enabweichungen und Verbrauchsabweichungen nach globalen Ursachenkomplexen und Verantwortungsbereichen untersucht werden.

\section{Ergänzung der periodenbezogenen Betriebsplankostenrechnung durch eine Online-Kennziffernrechnung}

Mit der Produktionsautomatisierung steigt der Anteil der Vorlaufkosten und oberflächliche Planungsansätze berücksichtigen nur einen relativ geringen Anteil aller Kosten als entscheidungsrelevant. Die Betriebsplankosten- und -erfolgsrechnung erfasst alle wesentlichen Kosten- und Erlösabhängigkeiten eines Unternehmens und macht den Periodenerfolg von Planungsalternativen berechenbar. Die zunehmende Anlagenintensität und veränderte Lohnformen erfordern eine operative, zeitaktuelle Kennzahlenrechnung zur laufenden Planung und Überwachung des Produktionsgeschehens. Bei einer hohen Anlagenintensität und Automatisierung des Produktionsprozesses können tägliche Abweichungen zwischen Vorgabe- und Istwerten erfasst, analysiert und Verantwortlichkeiten zugeordnet werden. Eine Online-Erfassung von Kosten, Erlösen und Mengen- und Zeitgrö{\ss}en legt Ursachen von Abweichungen offen und motiviert die Akteure im Produktions- und Absatzbereich zu Verhaltensänderungen. In einer betriebsplankosten- und -erfolgsrechnung wird der Grundgedanke um eine weitere zeitliche Differenzierung der Einflussgrö{\ss}en ergänzt, indem die zeitbezogen unterschiedliche Verfügbarkeit von Faktoreinsätzen und Absatzleistungen in den Einflussgrö{\ss}enfunktionen Berücksichtigung findet. Eine operative Kennzahlenrechnung ermöglicht schicht-, tages-, wochen- oder monatsbezogene Planungs- und Kontrollmöglichkeiten für Produktion und Absatz. Durch eine automatisierte Betriebsdaten- und Vertriebsdatenerfassung können tatsächliche Kosten und Erlöse sowie mengen- und zeitbezogene Istkennzahlen ermittelt und mit den entsprechenden Sollgrö{\ss}en verglichen werden, um Störungen und Unwirtschaftlichkeiten im Prozessablauf ohne Zeitverzug erkennbar zu machen. Über ein Online-Informationssystem im Produktions- und Vertriebsbereich können  Benutzer schicht-, tages-, wochen- und monatsbezogene Plan-/Soll- und/oder Soll-/Istvergleiche durchführen. Das Online-Informationssystem sollte mit der üblichen Monatsrechnung verbunden sein, die alle Erlös- und Kostengrö{\ss}en sowie betriebliche Kennziffern umfasst.

% \begin{figure}[H]
%  \centering
%  \includegraphics[height=5.82cm]{Bilder/REST_Rest.png}
%  \caption[REST Designprinzipien]{REST Designprinzipien. Abgerufen von \cite{fielding_architectural_2000} am 05.07.2023.}
%  \label{fig:iso_norm}
% \end{figure}

\chapter{Vergleich der Einkommensverteilungen anhand mehrerer Indikatoren}

Im Folgenden soll die Einkommensverteilung in Deutschland mit der weltweiten Einkommensverteilung anhand mehrerer Indikatoren verglichen werden. Die Indizes, die hierfür verwendet werden, sind das Verhältnis des Durchschnittseinkommens der reichsten 10\% zum Durchschnittseinkommens der ärmsten 50\%, der Gini-Koeffizient und der Bevölkerungsanteil mit einem Einkommen unterhalb der Armutsgrenze. Durch die Auswertung dieser Kennzahlen soll ein umfassendes Verständnis über die Einkommensverteilung Deutschlands im weltweiten Vergleich geschaffen werden.
\section{T10/ B50 Einkommensverhältnis}

Zuerst wird das gewichtete Verhältnis des Durchschnittseinkommens der reichsten 10\% zum Durchschnittseinkommen der ärmsten 50\% in Deutschland und auf globaler Ebene von 1900 bis 2020 betrachtet (Im Folgenden als ''T10/B50'' abgekürzt). Es gibt an, wie viel die reichsten 10\% im Vergleich zu den ärmsten 50\% verdienen gewichtet nach der absoluten Größe der Gruppen. Je höher der Wert, desto ungleicher sind die Einkommen verteilt. \footcite[Vgl.][S. 31]{wir_2022}

\begin{figure}[H]
    \centering
    \includegraphics[height=8.15cm]{Bilder/T10B50-Ratio.png}
    \caption[T10/B50-Verhältnis, Deutschland und global, 1900-2020]{gewichtetes Verhältnis des Durchschnittseinkommens der reichsten 10\% zum Durchschnittseinkommen der ärmsten 50\% in Deutschland und auf globaler Ebene von 1900 bis 2020. Eigene Darstellung und Berechnung. Daten abgerufen von \cite[][, S.55, 195]{wir_2022} am 01.03.2024.}
    \label{fig:iso_norm}
\end{figure}

\section{Gini-Koeffizient}

Der zweite herangezogene Indikator ist der Gini-Koeffizient. Er ist ein Maß für die (Un-)Gleichverteilung von Einkommen und Vermögen. Er kann Werte zwischen 0 und 1 annehmen, wobei 0 für eine vollkommene Gleichverteilung und 1 für eine vollkommene Ungleichverteilung steht. \footcite[Vgl.][]{gini_definition_diw_2024}

\begin{figure}[H]
    \centering
    \includegraphics[height=8cm]{Bilder/Gini-Koeffizient2.png}
    \caption[Gini-Koeffizient, Deutschland und global, 1995-2020]{Gini-Koeffizient für Deutschland und auf globaler Ebene von 1995 bis 2020. Eigene Darstellung. Daten abgerufen von \cite[][, S.56 (global)]{wir_2022} und \cite[][(Deutschland)]{bmas_arb_gini_2020} am 01.03.2024.}
    \label{fig:iso_norm}
\end{figure}

\section{Bevölkerungsanteil unterhalb der Armutsgrenze}

Als letzte Kennzahl wird der Anteil der Bevölkerung mit einem Einkommen unterhalb der Armutsgrenze betrachtet. In Deutschland ist die Armutsrisikoquote bei 60\% des Medianeinkommens angesetzt. \footcite[Vgl.][]{bmas_arb_armutsrisikoquote_2023} Für die globale Betrachtung werden die jeweiligen Einkommensgrenzen, die jedes Land für sich definiert hat, verwendet. \footcite[Vgl.][]{wb_armutsquote_global_2022}

\begin{figure}[H]
    \centering
    \includegraphics[height=6.9cm]{Bilder/Armutsgrenze2.png}
    \caption[Bevölkerungsanteil unterhalb der Armutsgrenze, Deutschland und global, 2000-2020]{Bevölkerungsanteil mit einem Einkommen unterhalb der Armutsgrenze in Deutschland und auf globaler Ebene von 2000 bis 2020. Eigene Darstellung und Berechnung. Daten abgerufen von \cite[][(global)]{wb_armutsquote_global_2022} und \cite[][(Deutschland)]{bmas_arb_armutsrisikoquote_2023} am 01.03.2024.}
    \label{fig:iso_norm}
\end{figure}


\chapter{Schlussbetrachtungen}

\section{Bewertung des Ansatzes}

% Die BPKER ist ein wichtiges Werkzeug für die Planung und Überwachung von Produktion und Absatz. Die Methode berücksichtigt alle relevanten Einflussfaktoren auf den Periodenerfolg, was sie zu einem detaillierten Modell für präzise Prognosen und Kontrolle macht.

% Ein vorteilhaftes Merkmal dieses Modells ist die Möglichkeit der taggenauen Planung. Dies kommt speziell Branchen mit sich schnell ändernden Produktions- und Absatzbedingungen zugute, wie zum Beispiel der Montanindustrie (Bergbau-, Stahlindustrie), in der sich externe Bedingungen schnell und erheblich ändern können. La{\ss}manns Methode hält dank der angelegten Flexibilität mit den Veränderungen Schritt und ermöglicht dadurch bessere Vorhersagen und Anpassungen.

% Eine weitere Stärke des BPKER liegt in ihrer Integration von Kostenrechnung und wesentlichen Planungssystemen. Der Ansatz führt zu einer hollistischen Darstellung der variablen und festen Kosten sowie der Modulebene Kostenanalyse. In anderen Modellen, wie der Grenzplankostenrechnung, werden Kosten- und Erlösrechnung oft getrennt behandelt, wodurch ein Teil der Information verloren gehen kann.

% Dennoch gibt es auch einige Nachteile bei der Verwendung der BPKER. Eines der Hauptprobleme ist der hohe Aufwand, der mit der Durchführung verbunden ist. Idealerweise wird das Modell möglichst oft mit allen relevanten Daten aktualisiert, um die genauesten Vorhersagen und Überwachungen für Produktion und Absatz zu ermöglichen. In der Praxis kann dieser Aktualisierungsprozess jedoch aufgrund der erforderlichen Ressourcen und des hohen Detailgrades zeitaufwändig und kostspielig sein.

% Ein weiterer Nachteil des Modells ist seine Komplexität. Die Berücksichtigung aller Einflussfaktoren führt zu einem komplexen Modell, das spezielles Fachwissen für seine Umsetzung und Überwachung erfordert. Das hindert kleinere Unternehmen am Zugang. Die Verwendung des Modells ist auch dadurch eingeschränkt, dass es auf die spezifischen Produktions- und Absatzbedingungen der Montanindustrie zugeschnitten ist.

% Au{\ss}erdem, obwohl La{\ss}manns Modell bei gro{\ss}en Unternehmen in der Montanindustrie implementiert wurde, wird die Methode insgesamt nicht so breit akzeptiert und angewandt wie andere Modelle (Grenzplankostenrechnung), woran höchstwahrscheinlich auch die höheren Ressourcenanforderungen und die grö{\ss}ere Komplexität des Modells schuld sein könnten.

% Zusammenfassend lässt sich sagen, dass die BPKER ein effektives Werkzeug zur detaillierten Planung und Überwachung von Produktion und Absatz ist. Es ermöglicht genaue Prognosen und bietet einen umfassenden Überblick über die Kosten und Erlöse. Allerdings sind die Ressourcenanforderungen und die Komplexität des Modells erhebliche Hindernisse für seine Anwendung, insbesondere in kleineren Unternehmen. Daher muss jedes Unternehmen sorgfältig abwägen, ob die Vorteile die potenziellen Nachteile überwiegen.

Die BPKER ist ein wichtiges Instrument für die Planung und Kontrolle von Produktion und Absatz. Sie bietet ein detailliertes und präzises Modell, welches alle relevanten Einflussfaktoren für den Geschäftserfolg einer Periode berücksichtigt. Besonders vorteilhaft daran ist die Möglichkeit einer taggenauen Planung, was vor allem für Branchen wie die Montanindustrie von Bedeutung ist, in denen sich Produktion und Absatzbedingungen schnell ändern können.

Die Methode vereint zudem die Kostenrechnung mit wesentlichen Planungssystemen. Dies ermöglicht eine ganzheitliche Darstellung und Analyse der variablen und festen Kosten.

Doch die BPKER weist auch erhebliche Nachteile auf. Durch ihre Komplexität und den hohen Aufwand, sie aktuell zu halten, ist sie sehr ressourcenintensiv. Diese Faktoren sind vor allem für kleinere Unternehmen, die möglicherweise nicht über ausreichend Ressourcen und Fachwissen verfügen, abschreckend. Zudem ist sie speziell auf die Bedürfnisse der Montanindustrie zugeschnitten, was ihre Anwendung in anderen Sektoren limitieren kann. Diese Faktoren sind auch der Grund, weshalb sich der Ansatz nie in der breiten Masse der Unternehmen durchsetzen konnte und keine praktische Relevanz erlangt hat. \footcite[Vgl.][(S. 292ff)]{franz2001beitrag}

Resümierend lässt sich sagen, dass die BPKER ein effektives und vielseitiges Werkzeug ist. Sie bietet einen umfassenden Überblick über Kosten und Erlöse und ermöglicht genaue Prognosen. Jedoch hat er sich aufgrund von Akzeptanzproblemen in der Wirtschaft aufgrund seiner Komplexität und seiner Aufwandes nie durchsetzen könne und ist sowohl in der Theorie als auch in der Praxis leider wieder in Vergessenheit geraten. \footcite[Vgl.][(S. 292ff)]{franz2001beitrag}

% \section{Reflexion der Arbeit und Ausblick}

% Bei der Untersuchung der BPKER wurden sowohl deren Potenziale als auch Herausforderungen deutlich. Es wurde veranschaulicht, wie dieses Modell durch die Berücksichtigung relevanter Einflussgrö{\ss}en und die detaillierte Darstellung von Produktions- und Absatzbedingungen eine umfassende Struktur für fundierte Unternehmensentscheidungen bietet. Gleichzeitig zeigte sich die Herausforderung der hohen Komplexität und der nicht zu vernachlässigbaren Ressourcen, die für die Implementierung dieses Ansatzes erforderlich sind.

% Die Anwendung des Modells auf die HobelKunstwerkstatt GmbH beleuchtete die potenzielle Realität einer solchen Anwendung. Es wurden wichtige Erkenntnisse darüber gewonnen, wie das Modell in der Praxis funktioniert und welche spezifischen Anforderungen und Hürden bei der Anwendung auftreten können. Zudem wurde deutlich, dass eine individuelle Anpassung an die Besonderheiten eines Unternehmens oder einer Branche wesentlich für den erfolgreichen Einsatz des Modells ist.

% In Bezug auf zukünftige Perspektiven und die Weiterentwicklung der BPKER besteht sicherlich ein erheblicher Forschungsbedarf. Jedoch bestehen im Hinblick auf die bereits weitreichende Verbreitung anderer Modelle und die fast schon in Vergessenheit geratene BPKER Zweifel daran, dass das Modell tatsächlich weiterentwickelt wird. Es wäre um es praktikabel einsetzbar zu machen wichtig, Methoden zu entwickeln, um die Effizienz des Modells zu erhöhen. In diesem Zusammenhang könnten Ansätze zur Integration von Technologien wie künstlicher Intelligenz und Data-Mining-Verfahren eine interessante Perspektive bieten. Zudem müsste das Modell für eine breite Pallette an Industrien angepasst werden, um ein Fundament für eine breite Akzeptanz zu schaffen. 

% Zusammenfassend bietet die BPKER eine detaillierte und umfassende Methode zur Planung und Kontrolle von Betriebsprozessen. Obwohl sie Herausforderungen mit sich bringt, bietet sie zugleich umfangreiche Möglichkeiten für die Betriebswirtschaft. Es bleibt spannend, wie sich die Forschung und Anwendung dieses Modells in der Zukunft weiterentwickeln werden.

% Durch die Untersuchung konnten Stärken und Schwächen der BPKER ermittelt werden. Das Modell bietet durch seine genaue Berücksichtigung verschiedener Einflussgrö{\ss}en eine fundierte Basis für Unternehmensentscheidungen. Gleichzeitig zeigt sich der hohe Ressourcenbedarf und die Komplexität als Herausforderung. Die Anwendung des Modells auf die HobelKunstwerkstatt GmbH hat die praktischen Anforderungen und Hindernisse beim Einsatz hervorgehoben und gezeigt, dass eine individuelle Anpassung an die spezifischen Anforderungen eines Unternehmens wichtig ist.

% Auf dem Gebiet der BPKER ist sicher noch gro{\ss}er Forschungsbedarf. Trotz der breiten Verwendung anderer Modelle und der marginalen Nutzung der BPKER gibt es Möglichkeiten zur Verbesserung, insbesondere durch die Implementierung technischer Lösungen wie künstlicher Intelligenz und Data-Mining. Eine breitgefächerte Industrieanpassung könnte zudem zur Akzeptanz des Modells beitragen.

% Insgesamt bietet die BPKER eine umfängliche Methode zur Planung und Kontrolle von Betriebsprozessen. Die weitere Entwicklung von Forschung und Anwendung dieses Modells bleibt ein interessanter Aspekt für die Zukunft.

% ---- Literaturverzeichnis
\cleardoublepage
\renewcommand*{\chapterpagestyle}{plain}
\pagestyle{plain}
%\pagenumbering{Roman}                   % Römische Seitenzahlen
%\setcounter{page}{\numexpr\value{savepage}+1}

\printbibliography[title = Literaturverzeichnis]

% ---- Anhang
\appendix
%\clearpage
%\pagenumbering{Roman}  % römische Seitenzahlen für Anhang

\chapter{Anhang}

% \section{Fragebögen Experteninterviews}

% \subsection{Workflows} % \label{FB_Workflows}

abc

%\includepdf[pages = -]{Literatur/bgPF_Wiki.pdf}

\newpage
\end{document}
