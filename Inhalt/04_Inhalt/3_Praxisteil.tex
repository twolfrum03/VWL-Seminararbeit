\chapter{Praktischer Teil}


\section{Anwendung auf die HobelKunstwerkstatt GmbH}

\subsection{Unternehmensporträt}

Um die BPKER auf ein fiktives Unternehmensszenario anzuwenden, soll zuerst das betrachtete Unternehmen kurz skizziert werden. Die HobelKunstwerkstatt GmbH ist eine kleine Schreinerei mit zehn Mitarbeitern, die hochwertige Designermöbel herstellt. Das seit 90 Jahren in dritter Generation familiengeführte Unternehmen hat sich hauptsächlich auf zwei Produkte spezialisiert: Tische und Stühle aus Eichenholz. Da die Produkte im Premiumsegment angesiedelt sind, hat das Unternehmen im letzten Jahr einen Umsatz von 2,4 Mio. {\euro} erwirtschaftet. Hiervon entfällt der Großteil auf die beiden Hauptprodukte und ein kleiner Teil auf den Verkauf von Pflegeprodukten und die Aufbereitung von älteren Möbeln.

\subsection{Betriebs- und Absatzmodell}

\subsubsection{Betriebsmodell}

Ein Betriebsmodell definiert, wie ein Unternehmen seine Geschäftsprozesse und -strukturen für eine effiziente Wertschöpfung organisiert. Im Fall der HobelKunstwerkstatt GmbH sieht das Betriebsmodell folgendermaßen aus:

\begin{itemize}
    \item \textbf{Beschaffung:} Die HobelKunstwerkstatt arbeitet mit mehreren Lieferanten aus der Umgebung zusammen, um lokal nachhaltiges Holz zu beschaffen. Alle anderen Hilfsstoffe werden vom Großhandel bezogen. 
    \item \textbf{Produktion:} Das Unternehmen produziert die beiden Hauptprodukte in der eigenen Produktionsstätte. Der Produktionsprozess besteht aus dem Verarbeiten des Rohmaterials in die Einzelteile der Produkte, den Zusammenbau der Einzelteile, dem Veredeln des Produkts und abschließend der Qualitätskontrolle.
    \item \textbf{Verwaltung/ Vertrieb:} Das Unternehmen hat eine Bürokraft, die sich zusammen mit dem Inhaber um sämtliche Verwaltungsaufgaben kümmert. Auf die Vertriebswege wird im Absatzmodell noch näher eingegangen.
\end{itemize}

\subsubsection{Absatzmodell}

Ein Absatz- oder Vertriebsmodell definiert wie ein Unternehmen seine Produkte an seine Kunden verkauft. Das Absatzmodell der HobelKunstwerkstatt GmbH wird im Folgenden skizziert:

\begin{itemize}
    \item \textbf{Direktverkauf:} Der Hauptumsatz des Unternehmens wird über den Direktverkauf erwirtschaftet. Das Unternehmen hat einen Showroom, in dem die angebotetenen Tische und Stühle in verschiedenen Varianten, sowie die Anpassungsmöglichkeiten für die Kunden präsentiert werden.
    \item \textbf{Online-Shop:} Zudem betriebt die HobelKunstwerkstatt einen Online-Shop, über den Kunden sich ihr Wunschprodukt digital konfigurieren und bestellen können. Zudem werden hierüber Pflegeprodukte vertrieben. Aufgrund der hochpreisigen Produkte ist der Online-Shop aber nur für einen kleinen Teil des Umsatzes verantwortlich.
\end{itemize}

\subsection{Einflussgrößen auf das Betriebs- und Absatzmodell}

\subsubsection{Primäre Einflussgrößen}

Primäre Einflussgrößen sind Einflüsse, die einen direkten und wesentlichen Einfluss auf das Betriebs- und Absatzmodell haben. Änderungen dieser Einfussgrößen resultieren meistens in erheblichen Schwankungen \zB der Produktkosten oder Verkaufserlöse. Deshalb ist es besonders wichtig diese konstant zu überwachen, um ggf. frühzeitig Anpassungen in den betroffenen Faktoren vornehmen zu können.

Im Betriebsmodell sind die Rohstoff- und Arbeitskosten die wichtigsten primären Einflussgrößen. Diese machen zusammen beim Produkt Tisch 81,25 \% (Rohstoffe: 31,25 \% bzw. Arbeit: 50 \%) und beim Produkt Stuhl 83,33 \% (Rohstoffe: 33,33 \% bzw. Arbeit: 50 \%) aus. Das ist ein typisches Phänomen, da Deutschland ein Hochlohnland ist und somit der Arbeitslohn meist einen Großteil der Produktkosten ausmacht. Der hohe Einfluss der Rohstoffkosten ist hingegen eher durch den Fokus des Unternehmens, das verwendete Holz lokal und auch nachhaltigen Quellen zu beschaffen, zu erklären.

Im Absatzmodell sind der Preispunkt der Produkte und die Marktnachfrage die bedeutendsten primären Einflussgrößen. Der Preispunkt hat einen erheblichen Einfluss auf die Verkaufszahlen und somit auf den Umsatz bzw. das Betriebsergebnis des Unternehmens. Gerade weil das Unternehmen Produkte im oberen Premiumsegment produziert, muss dieser sorgfältig gewählt werden, um die Balance zwischen Verkaufszahlen und Gewinnmarge zu halten. Auch die Marktnachfrage ist ein wichtiger Faktor, da die von dem Unternehmen verkauften Produkte eine sehr lange Lebenszeit haben und von den Kunden nicht nach einigen Jahren schon wieder erneuert werden. Jedoch ist dieser vom Unternehmen nicht komplett beeinflussbar, da die wirtschaftliche Situation außerhalb seiner Kontrolle liegt und Kunden in schwierigen konjunkturellen Rahmenbedingungen Luxuskäufe eher noch zu Gunsten eines preiswerteren Produkts überdenken. 

\subsubsection{Sekundäre Einflussgrößen}

Sekundäre Einflussgrößen sind Faktoren, die das Betrieb- und Absatzmodell zwar beeinflussen, dies aber in eher indirekt und weniger stark. Zudem wirken sich diese Einflussgrößen eher über Umwege bzw. in Kombination mit anderen Faktoren auf die Modelle aus. Somit sollten diese zwar auch regelmäßig überwacht werden, jedoch nicht genauso wie die primären Einflussgrößen im Zentrum stehen.

Im Betriebsmodell wären sekundäre Einflussgrößen die Effizienz der Produktionsprozesse und die Qualität der Rohstoffe. Der erste Faktor beeinflusst, wie schnell und wirtschaftlich das Unternehmen produziert. Das wirkt sich somit einerseits intern auf die Zahlen des Unternehmens und andererseits extern auf Lieferzeiten und somit auch die Kundenzufriedenheit aus. Die Rohstoffqualität wirkt sich direkt auf die Qualität der Produkte aus und somit auch indirekt auf die Kundenzufriedenheit. Qualität ist gerade bei hochpreisigen Produkten ein wichtiger Faktor.

Als sekundäre Einflussgrößen im Absatzmodell können das Marketing und der Kundenservice benannt werden. Marketing ist für die HobelKunstwerkstatt ein Hebel um die Bekanntheit der Marke und somit auch die Nachfrage nach den eigenen Produkten zu beeinflussen. Gerade im Premiumsegment ist auch der Kundenservice ein sehr wichtiger Faktor, da Kunden beim Kauf eines so teuren Produkts meist Dinge wie Lieferung und Montage und unkomplizierte Hilfe bei Problemen erwarten. Somit können auch Weiterempfehlungs- und Wiederverkaufsraten gesteigert werden.

\subsection{Einflussgrößenfunktionen}

Im Folgenden sollen noch die Auswirkungen der Einflussgrößen auf verschiedene Aspekte des Betriebs- und Absatzmodells in einer Strukturmatrix veranschaulicht werden.

\subsubsection{Betriebsmodell}

\begin{table}[h]
    \centering
    \begin{tabular}{|c|c|c|c|}
      \hline
      Einflussgröße & Produktionskosten & Produktqualität & Produktionskapazität \\
      \hline
    \end{tabular}
    \caption{Strukturmatrix für das Betriebsmodell}
  \end{table} 

\subsubsection{Absatzmodell}

\begin{table}[h]
    \centering
    \begin{tabular}{|c|c|c|c|}
      \hline
      Einflussgröße & Verkaufszahlen & Umsatz/ Periondenergebnis & Kundenzufriedenheit \\
      \hline
    \end{tabular}
    \caption{Strukturmatrix für das Absatzmodell}
  \end{table} 

\subsection{Betriebsplankosten- und -erfolgsrechnung}

Die Kostenstruktur für die Herstellung dieser Produkte setzt sich vereinfachend aus Materialkosten, Arbeitskosten und dem Deckungsbeitrag für alle sonstigen Kosten zusammen. In der folgenden Tabelle werden diese Kosten genauer aufgeschlüsselt:

\begin{table}[h]
    \centering
    \label{tab:Kostenstruktur_Produkte}
    \begin{tabular}{|c|c|}
      \hline
      Tisch & Stuhl \\
      \hline
      Arbeitskosten: 4.000 {\euro} & Arbeitskosten: 600 {\euro} \\
      Materialkosten: 2.500 {\euro} & Materialkosten: 400 {\euro} \\
      Deckungsbeitrag: 1.500 {\euro} & Deckungsbeitrag: 200 {\euro} \\ 
      \hline
      Kosten insgesamt: 8000 {\euro} & Kosten insgesamt: 1.200 {\euro} \\
      \hline
      Nettoverkaufspreis: 9.600 {\euro} & Nettoverkaufspreis: 1.440 {\euro} \\
      \hline
      Bruttoverkaufspreis: 11.424 {\euro} & Bruttoverkaufspreis: 1713,60 {\euro} \\
      \hline
    \end{tabular}
    \caption{Kostenstruktur der Produkte}
  \end{table}

Für die konkrete Anwendung soll jetzt angenommen werden, dass das Unternehmen im Monat zehn Tische und 40 Stühle fertigt. Da die Gewinnmarge bei beiden Produkten 20 \% und die Kunden aufgrund der abgestimmten Optik meist einen Tisch zusammen mit vier Stühlen kaufen, würde es auch nicht sinnvoll sein, die vorhandenen Produktionskapazitäten zugunsten des Produkts Tisch zu verschieben, auch wenn bei diesem der Gewinn absolut höher ausfallen würde.

Mit diesen Informationen kann die BPKR verwendet werden, um die Gesamtkosten für die Produktion im nächsten Monat zu berechnen.:

\[ Kosten Tische: 8.000 EUR * 10 = 80.000 EUR \]
\[ Kosten Stühle: 1.200 EUR * 40 = 48.000 EUR \]
\[ Gesamtkosten: 80.000 EUR + 48.000 EUR = 128.000 EUR \]

Die BPKR bietet jedoch nicht nur eine Momentaufnahme der Kostenstruktur, sondern auch eine Grundlage für die BPER. Hierfür sollen im Folgenden zuerst die Gesamterlöse des nächsten Monats berechnet werden.

\[ Erlöse Tische: 9.600 EUR * 10 = 96.000 EUR \]
\[ Erlöse Stühle: 1.440 EUR * 40 = 57.600 EUR \]
\[ Gesamterlöse: 96.000 EUR + 57.600 EUR = 153.600 EUR \]

Mit den Ergebnissen BPKR lässt sich nun mittels der BPER das Betriebsergebnis durch Abziehen der Kosten von den Erlösen ermitteln:

\[ 153.600 EUR - 128.000 EUR = 25.600 EUR \]

Somit würde das Unternehmen im nächsten Monat einen Gewinn von 25.600 {\euro} erzielen, wenn alle Produktions- und Verlaufspläne genauso umgesetzt werden.

Es ist wichtig zu beachten, dass sowohl die Betriebsplankostenrechnung als auch die Betriebsplanerfolgsrechnung von verschiedenen Einflussgrößen beeinflusst werden. Diese Einflussgrößen können in einer Strukturmatrix dargestellt werden, die die Beziehungen zwischen den verschiedenen Faktoren in den Modellen zeigt. In unserer Strukturmatrix für das Betriebsmodell haben wir beispielsweise Rohstoffkosten, Arbeitskosten, Effizienz der Produktionsprozesse und Qualität der Rohstoffe als Einflussgrößen identifiziert. Im Absatzmodell haben wir Preispunkt, Marktnachfrage, Marketing und Werbung, sowie Kundenservice als Einflussgrößen identifiziert.

Zusammenfassend lässt sich sagen, dass die Betriebsplankostenrechnung und die Betriebsplanerfolgsrechnung wertvolle Werkzeuge für die Planung und Steuerung der Produktion und des Verkaufs in einem Unternehmen sind. Sie ermöglichen es dem Unternehmen, informierte Entscheidungen zu treffen und seine Ressourcen effizient zu nutzen, um seine Geschäftsziele zu erreichen. Dabei spielen verschiedene Einflussgrößen eine entscheidende Rolle und sollten bei der Planung und Steuerung berücksichtigt werden.