\chapter{Einleitung}

Die Planung und Kontrolle von Produktions- und Absatzaktivitäten ist ein zentrales Element moderner Betriebswirtschaft und Gegenstand zahlreicher theoretischer Ansätze. Einer dieser Ansätze ist das Modell der Betriebsplankosten- und Betriebsplanerfolgsrechnung von Gert La{\ss}mann von 1968. Dieses Modell bietet einen detaillierten und gründlichen Rahmen für die Berücksichtigung aller zentralen Einflussfaktoren auf den Periodenerfolg und hebt sich durch seine Akzentuierung von intensiver Planung und Überwachung ab, insbesondere in sich Branchen mit sich schnell verändernden Rahmenbedingungen, wie der Montantindustrie. Vor diesem Hintergrund zielt diese Arbeit darauf ab, die theoretischen Konzepte von Gert La{\ss}mann umfassend vorzustellen und sie auf ein konkretes Praxisbeispiel anzuwenden. Hierfür wird die fiktive HobelKunstwerkstatt GmbH als Fallstudie herangezogen. Die Arbeit verfolgt das Ziel, durch die theoretische Betrachtung und praktische Anwendung der Betriebsplankosten- und Betriebsplanerfolgsrechnung detaillierte Einblicke zu gewinnen und eine fundierte Bewertung des Modells im Kontext realer Geschäftsszenarien zu ermöglichen. Abschlie{\ss}end wird eine Bewertung des Ansatzes vorgenommen, die seine Stärken und möglichen Schwachstellen dezidiert betrachtet.
