\chapter{Theoretische Grundlagen}

\section{Grundlagen der Betriebsplankosten- und Betriebsplanerfolgsrechnung}

\subsection{Betriebs- und Absatzmodelle}

Die Betriebsplankosten- und -erfolgsrechnung ist eine Weiterentwicklung der flexiblen Plankostenrechnung und Deckungsbeitragsrechnung. Sie verknüpft das betriebliche Rechnungswesen direkt mit den Planungs- und Überwachungsprozessen in Absatz und Produktion. Alle wichtigen Einflussgrö{\ss}en auf den Periodenerfolg müssen in Planung und Abrechnung erfasst werden. Als materielle Basis dienen Betriebs- und Absatzmodelle, die Einflussgrö{\ss}enbeziehungen in mathematischen Funktionen abbilden. Nur solche Einflussgrö{\ss}enbeziehungen werden berücksichtigt, die für die Planung und Überwachung von Produktion und Absatz sowie für die Produktkalkulation und Preisbeurteilung entscheidend sind.

Betriebsmodelle basieren auf engineering production functions oder Prozessmodellen, die zur Steuerung technologischer Produktionsabläufe eingesetzt werden. Beispiele dafür sind Hochofenprozessmodell, Stahlwerksprozessmodell oder verfahrenstechnische Prozessmodelle in der Chemieindustrie. Diese Prozessmodelle sind oft zu feingliedrig und komplex für betriebswirtschaftliche Aufgaben, insbesondere hinsichtlich der Wirtschaftlichkeit des Rechnungswesens. Daher müssen aus ihnen die betriebswirtschaftlich relevanten Einflussgrö{\ss}enbeziehungen abgeleitet werden. Einige Modelle beinhalten nichtlineare Einflussgrö{\ss}enfunktionen, bei denen oft eine (abschnittsweise) lineare Approximation für betriebswirtschaftliche Planungsansätze ausreicht. In einer produktionstheoretischen Sicht basieren Betriebsmodelle auf den Input-Output-Modellen von Leontief (1953), Betriebsmatrizen von Pichler (1961) und Verbrauchsfunktionen von Gutenberg (1983).

Absatzmodelle basieren in der Regel auf der Analyse vergangener Absatzprozesse und Projektionen zukünftiger Absatzaktivitäten. Absatzleistungsarten und entsprechend bewerteter Erlösarten transparent machen. Auf diese Basis können marktsegmentspezifische Absatzmodelle erstellt werden. Absatzmodelle stellen die verschiedenartigen Absatzleistungen in Abhängigkeit von ihren Haupteinflussgrö{\ss}en dar und unterstützen Prognosen über zukünftige Absatzentwicklungen. Für die Ableitung von Absatz- und Erlösplänen sowie spezifischen Vorgaben im Vertriebsbereich müssen zusätzlich die Einflüsse von geplanten Absatzaktivitäten und bereits vorhandenen Auftragsbeständen berücksichtigt werden. Die auf den Absatzmodellen basierende Planerlösrechnung hat mindestens ebenso hohe Bedeutung wie die Betriebsplankostenrechnung. Ohne Kenntnis der wichtigsten Erlöseinflussgrö{\ss}en und der im Absatz vorherrschenden Wirkungszusammenhänge ist eine fundierte Absatz- und Erlösplanung nicht möglich.

\subsection{Periodenerfolg als Lenkungsziel}

Wirtschaftliches Ziel von Unternehmen ist die Erreichung hoher Periodenerfolge. Die Betriebsplankosten- und -erfolgsrechnung basiert auf Betriebs- und Absatzmodellen und zielt auf die Ermittlung erfolgsoptimaler Produktions- und Absatzpläne ab. Die Betriebsplankosten- und -erfolgsrechnung dient zur Überwachung der Planumsetzung, differenziert nach den wichtigsten Erfolgskomponenten. Auf Basis von Einflussgrö{\ss}enfunktionen werden für alternative Produkt- und Absatzprogramme, verfügbare Produktions- und Absatzbedingungen und/oder Faktoreinsatzzusammensetzungen die zu erwartenden Periodenkosten und -erlöse ermittelt. Der Produktions- und Absatzplan kann durch lineare Programmierung optimiert werden, sofern ausreichende Informationen vorhanden sind. Im Gegensatz zur flexiblen Plankosten- und Deckungsbeitragsrechnung dient die Betriebsplankosten- und -erfolgsrechnung gleichzeitig der periodenbezogenen Produktions- und Absatzplanung sowie der Planung und Überwachung der periodenbezogenen Produktionskosten und Absatzerlöse. Mit der Betriebsplankosten- und -erfolgsrechnung können auch Stückkosten und -erlöse je Produktions- und Absatzbereich ermittelt werden. Die Ursachen für Plan-Ist-Abweichungen können detaillierter ermittelt werden als in der flexiblen Plankostenrechnung. Der Ansatz kann auch alle notwendigen Dokumentationsanforderungen erfüllen.

\section{Aufbau der Betriebsplankosten- und Betriebsplanerfolgsrechnung}

\subsection{Ermittlung von Einflu{\ss}grö{\ss}enfunktionen}

Der Aufbau einer Betriebsplankosten- und -erfolgsrechnung beginnt mit der Bestimmung wesentlicher Kosten- und Erlöseinflussgrö{\ss}en, die als Ursachen von Kostenverbräuchen oder betrieblichen Faktoreinsätzen angesehen werden können. Die Disponibilität der Einflussgrö{\ss}en ist ein wichtiges Kriterium, dabei gibt es Einflussgrö{\ss}en, die frei von der Betriebsleitung verfügbar sind und solche, die extern bestimmt sind. Neben primären Einflussgrö{\ss}en, die von der Betriebsleitung oder der "Umwelt" bestimmt werden, gibt es sekundäre Einflussgrö{\ss}en, die als Zwischenergebnisse in der betrieblichen Einflussgrö{\ss}enrechnung anfallen. Methoden zur Bestimmung der Wirkungsweise von Einflussgrö{\ss}en auf Faktoreinsätze und Absatzleistungen umfassen statistische und analytische Verfahren. Bei statistischen Verfahren werden aus vergangenen Istwerten Faktoreinsatz- und Absatzleistungsfunktionen abgeleitet, bei analytischen Verfahren werden diese Beziehungen auf der Grundlage theoretischer Studien und wissenschaftlicher Erkenntnisse festgelegt. Regelmä{\ss}ige Überprüfungen der Einflussgrö{\ss}enfunktionen sind notwendig, da der Gültigkeitsbereich der statistisch ermittelten Koeffizienten begrenzt ist und Veränderungen im Zeitverlauf eine Neubestimmung der Koeffizienten erfordern können. Ein Beispiel für eine konkrete Betriebsstoffeinsatzfunktion zeigt die Grundstruktur von Einflussgrö{\ss}enfunktionen eines Betriebsmodells.

\subsection{Strukturelemente von Betriebs- und Absatzmodellen}

Im ersten Schritt zur Erstellung einer Betriebsplankosten- und -erfolgsrechnung erfolgt die Bestimmung wesentlicher Kosten- und Erlöseinflussgrö{\ss}en. Einflussgrö{\ss}en, die als Ursachen für Kostengüterverbräuche, betriebliche Faktoreinsätze und Absatzleistungen betrachtet werden, bilden unabhängige Variablen. Einflussgrö{\ss}en können von Betriebsleitung frei verfügbar sein oder durch externe Faktoren bestimmt werden. Neben den primären Einflussgrö{\ss}en gibt es sekundäre Einflussgrö{\ss}en, die als Zwischenergebnisse in der betrieblichen Einflussgrö{\ss}enrechnung anfallen. Die Wirkungsweise von Einflussgrö{\ss}en auf ausgewählte Faktoreinsätze und Absatzleistungen kann durch statistische und analytische Verfahren bestimmt werden. Die Strukturmatrix stellt ein allgemeines Ordnungsschema für Vektoren und Matrizen dar, mit dem Produktions- und Absatzprozesse abgebildet werden. Die Strukturmatrix beinhaltet Vorgabegrö{\ss}en oder primäre Einflussgrö{\ss}en sowie abgeleitete Zwischenzielgrö{\ss}en und resultierende Kostengüterverbrauchs-Zielgrö{\ss}en. In den Feldern der Strukturmatrix sind Koeffizientenmatrizen untergebracht. Die letzte Zeile der Strukturmatrix enthält technologische oder durch Umwelt und Betriebsleitung bestimmte Einschränkungen. Ähnlich aufgebaut ist die Strukturmatrix eines Absatzmodells, wo im ersten Schritt das Absatzmengenprogramm, differenziert nach Qualitäten und/oder Abmessungen, bestimmt wird. Aufgrund der Unberechenbarkeit des Marktverhaltens ist meist das differenzierte Absatzprogramm selbst eine Vorgabe. Für die Monats- und Quartalsplanung von Absatz und Produktion bzw. Kosten und Erlösen können oft Vorgaben für den Absatzbereich aus vorhandenen Auftragsbeständen abgeleitet werden. Abschlie{\ss}end wird das Absatzprogramm, differenziert nach Vertriebswegen, sonstigen Dienstleistungen und anderen erlösbestimmenden Komponenten ermittelt.

\subsection{Verknüpfung von Betriebs- und Absatzmodellen zum Periodenerfolgsmodell}

Betriebs- und Absatzmodelle stellen Mengenbeziehungen zwischen Produktmengen und deren Haupteinflussgrö{\ss}en dar und haben den Charakter von Input-Output-Modellen. Durch multiplikative Verknüpfung mit Preisvektoren entstehen Planerlöse und Plankosten je Periode, differenziert nach Absatzleistungsarten und Kostenarten. Der Planperiodenerfolg wird als Differenz zwischen den Gesamtperiodenerlösen und -kosten berechnet. Bei der Verknüpfung aufeinanderfolgender Periodenmodelle sind Bestandszuführungen und -entnahmen von Halb- und Fertigfabrikaten zu beachten. Es können sowohl Alternativkalkulationen durchgeführt als auch Optimierungsrechnungen unter Einsatz linearer Programmierungsansätze angewendet werden.

\section{Einsatz der Betriebsplankosten- und -erfolgsrechnung im Unternehmenscontrolling bei Sorten- und Serienfertigung}

\subsection{Integrierte Produktions-, Absatz-, Kosten-, Erlös- und Erfolgsplanung}

Betriebs- und Absatzmodelle reflektieren Mengenbeziehungen zwischen Produktionsprogramm und Haupteinflussgrö{\ss}en, visualisiert durch Input-Output-Modelle. Die multiplikative Verknüpfung mit Preisvektoren bildet Planerlöse und Plankosten pro Periode, differenziert nach Absatzleistungs- und Kostenarten. Wechsel im Produktionsprogramm und Umfeld erfordern Anpassungen im Herstellungsprozess, beispielsweise Verfahrenswechsel, In- oder Au{\ss}erbetriebnahme von Anlagen, Variation von Fertigungslosgrö{\ss}en etc. Wirtschaftliche Kriterien für solche Entscheidungen sind die Erfolgswirkungen verschiedener Ma{\ss}nahmen. Zwei separate Schritte sind erforderlich: eine Mengenberechnung und eine Bewertungsrechnung. Mengenberechnung berechnet Erzeugniseinsatzstoffmengen und Planzeiten der Systeme aus dem Produktionsprogramm und sonstigen Einflussgrö{\ss}en. Bewertungsrechnung berechnet die Periodenkosten, indem sie Kostengüterbedarfsmengen mit entsprechenden Preiskategorien verknüpft. Es besteht eine zunehmende Differenz zwischen Plankosten und aktuellem Preisniveau im Jahresverlauf. Vor allem bei kurz- und mittelfristigen Anpassungen an Beschäftigungsschwankungen kommen Optimierungsansätze zum Einsatz. Alternativen im Produktionsvollzug und alternative Absatzmengen erfordern wirtschaftliche Beurteilung. Die Bedingungen für einen optimalen Erfolg können unter bestimmten Voraussetzungen bestimmt werden. Simulationen können die Einflüsse verschiedener Rabattmengen, Rabattsatzgestaltungen, Bonusgewährungen, und Vertriebswegealternativen auf Periodenerlös evaluieren. Periodenerfolgsmaximale Produktions- und Absatzprogramme können ermittelt werden, wenn sowohl Kosten- als auch Erlöseinflussgrö{\ss}en verfügbar sind.

\subsection{Produktions-, Absatz-, Kosten-, Erlös-, und Erfolgsüberwachung}

Die Abweichungsanalyse ist ein wesentlicher Bestandteil der kurzfristigen Erfolgsermittlung und -überwachung und kann auf jedem gewünschten Detailgrad durchgeführt werden. Die Aussagekraft der Analyse hängt von der gewählten Zielsetzung und dem Differenzierungsgrad der Betriebs- und Absatzmodelle ab. Die Analyse kann Ursachen und Verantwortlichkeiten für Abweichungen aufzeigen, beispielsweise in Bereichen wie Leistung, Produktion, Material, Arbeitszeit etc. Die Abweichungsanalyse gliedert die Differenz zwischen Plan- und Isterfolg nach Erlös- und Kostenabweichungen und unterscheidet jeweils nach Preis- und Mengenabweichungen. Planänderungen und Verbrauchsabweichungen werden in der Kostengütermengenüberwachung detailliert ermittelt und unterschieden. Es sind Anpassungen an veränderte Umstände erforderlich, die sich in veränderten Kostengütermengenabweichungen niederschlagen. Abweichungen von den Sollgrö{\ss}en sind zu berücksichtigen, wenn sie au{\ss}erhalb der statischen Streubreite liegen. Solche Abweichungen können entweder negativ (z.B. durch Materialfehler oder Fehlverhalten der Arbeitskräfte) oder positiv (z.B. durch technische Verbesserungen oder höhere Arbeitseffizienz) sein. Die auf Entscheidungen beruhenden Abweichungen ergeben sich durch den Vergleich von am Monatsanfang berechneten Plankosten mit den Sollkosten, die nach Monatsende berechnet wurden. Am Monatsende können auf Basis der tatsächlichen Produktionsmengen und -bedingungen die Sollkosten berechnet und den Istkosten gegenübergestellt werden. Nur wesentliche Abweichungen, die vom Empfänger des Berichts verursacht und zu verantworten sind, sollten ermittelt und dokumentiert werden.

\subsection{Produktkalkulation zur Ermittlung von Preisgrenzen und Produkterfolgen}

Die Erzeugnis- und Auftragskosten und -erlöse werden insbesondere für die Bildung von Angebotspreisen, zur Beurteilung von Marktpreisen, zur Bewertung von Lagerbeständen und zur Ermittlung von Produkterfolgen benötigt. Mit der Betriebsplankosten- und -erfolgsrechnung kann die Kostenträgerrechnung in jeder gewünschten Kostenabgrenzung vollzogen werden. Um Kostenträgerrechnung durchzuführen, müssen bestimmte Vorgaben für alle Freiheitsgrade eingesetzt werden. Unter Risikogesichtspunkten können auch Alternativkalkulationen durchgeführt werden. Aufgrund der Trennung zwischen Mengen- und Bewertungsrechnung in der Endphase der Kostenermittlung können die Auswirkungen von Preisänderungen auf die Herstellungskosten gezeigt werden. In analoger Form erlauben Absatzmodelle die Durchführung einer Erlöskalkulation zur Bestimmung von Brutto- und Nettoerlösen je Einheit des Absatzprogramms. Durch Zusammenführung von Erlös- und Kostenkalkulation lässt sich eine Erfolgskalkulation pro Erzeugniseinheit oder Auftrag aufbauen. Stückerfolgsgrö{\ss}en haben für die Betriebsplankosten- und -erfolgsrechnung nur eine sekundäre Bedeutung, da sie für die periodenbezogene Produktions- und Absatzplanung nicht benötigt werden.

\section{Konzeptionelle Unterschiede zwischen Betriebsplankosten- und -erfolgsrechnung und flexibler Plankostenrechnung}

Die flexible Plankosten- und Deckungsbeitragsrechnung geht davon aus, dass sich die variablen Kosten und Erlöse allein zu Produktmengenvariationen je Periode proportional verhalten. Alle anderen Kosten- und Erlöseinflussgrö{\ss}en werden in der flexiblen Plankosten- und Deckungsbeitragsrechnung im Zuge vorgelagerter Planungsprozesse festgelegt. Die Betriebsplankosten- und -erfolgsrechnung hingegen erfasst alle wesentlichen Kosten- und Erlösabhängigkeiten eines Unternehmens und ermöglicht die Berechnung des Periodenerfolgs von Planungsalternativen. In der flexiblen Plankostenrechnung ist die Ermittlung einer Vielzahl von alternativ-konstanten Grenzplankosten für jede zu erwartende Kombination der Einflussgrö{\ss}enwerte notwendig. Die flexible Plankosten- und Deckungsbeitragsrechnung eignet sich insbesondere für die Kalkulation, Planung und Kontrolle von Produktions- und Absatzprozessen, während die Betriebsplankosten- und -erfolgsrechnung die periodenbezogene Kosten- und Erlös- bzw. Erfolgsermittlung im Zusammenhang mit der kurz- bis mittelfristigen Planung und Kontrolle von Produktion und Absatz ermöglicht. Bei der Betriebsplankosten- und -erfolgsrechnung erfolgt eine strikte Trennung von flexibler Mengen- und Bewertungsrechnung mit laufender Preisaktualisierung, während in der flexiblen Plankostenrechnung die Mengen- und Bewertungsrechnung integriert sind. In der Betriebsplankosten- und -erfolgsrechnung können Abweichungen nach technischen Einzelursachen untergliedert und unter Berücksichtigung von Verantwortlichkeiten analysiert werden, während in der flexiblen Plankostenrechnung Beschäftigungs- bzw. Bezugsgrö{\ss}enabweichungen und Verbrauchsabweichungen nach globalen Ursachenkomplexen und Verantwortungsbereichen untersucht werden.

\section{Ergänzung der periodenbezogenen Betriebsplankostenrechnung durch eine Online-Kennziffernrechnung}

Mit der Produktionsautomatisierung steigt der Anteil der Vorlaufkosten und oberflächliche Planungsansätze berücksichtigen nur einen relativ geringen Anteil aller Kosten als entscheidungsrelevant. Die Betriebsplankosten- und -erfolgsrechnung erfasst alle wesentlichen Kosten- und Erlösabhängigkeiten eines Unternehmens und macht den Periodenerfolg von Planungsalternativen berechenbar. Die zunehmende Anlagenintensität und veränderte Lohnformen erfordern eine operative, zeitaktuelle Kennzahlenrechnung zur laufenden Planung und Überwachung des Produktionsgeschehens. Bei einer hohen Anlagenintensität und Automatisierung des Produktionsprozesses können tägliche Abweichungen zwischen Vorgabe- und Istwerten erfasst, analysiert und Verantwortlichkeiten zugeordnet werden. Eine Online-Erfassung von Kosten, Erlösen und Mengen- und Zeitgrö{\ss}en legt Ursachen von Abweichungen offen und motiviert die Akteure im Produktions- und Absatzbereich zu Verhaltensänderungen. In einer betriebsplankosten- und -erfolgsrechnung wird der Grundgedanke um eine weitere zeitliche Differenzierung der Einflussgrö{\ss}en ergänzt, indem die zeitbezogen unterschiedliche Verfügbarkeit von Faktoreinsätzen und Absatzleistungen in den Einflussgrö{\ss}enfunktionen Berücksichtigung findet. Eine operative Kennzahlenrechnung ermöglicht schicht-, tages-, wochen- oder monatsbezogene Planungs- und Kontrollmöglichkeiten für Produktion und Absatz. Durch eine automatisierte Betriebsdaten- und Vertriebsdatenerfassung können tatsächliche Kosten und Erlöse sowie mengen- und zeitbezogene Istkennzahlen ermittelt und mit den entsprechenden Sollgrö{\ss}en verglichen werden, um Störungen und Unwirtschaftlichkeiten im Prozessablauf ohne Zeitverzug erkennbar zu machen. Über ein Online-Informationssystem im Produktions- und Vertriebsbereich können  Benutzer schicht-, tages-, wochen- und monatsbezogene Plan-/Soll- und/oder Soll-/Istvergleiche durchführen. Das Online-Informationssystem sollte mit der üblichen Monatsrechnung verbunden sein, die alle Erlös- und Kostengrö{\ss}en sowie betriebliche Kennziffern umfasst.

% \begin{figure}[H]
%  \centering
%  \includegraphics[height=5.82cm]{Bilder/REST_Rest.png}
%  \caption[REST Designprinzipien]{REST Designprinzipien. Abgerufen von \cite{fielding_architectural_2000} am 05.07.2023.}
%  \label{fig:iso_norm}
% \end{figure}
