\chapter{Theoretische Grundlagen}

\section{Zusammenfassung des grundlegenden Artikels}

\subsubsection{Grundlagen der Betriebsplankosten- und -erfolgsrechnung (BPKER)}

Die BPKER, die das betriebliche Rechnungswesen mit den Planungs- und Überwachungsprozessen von Produktion und Absatz verknüpft, ist eine Weiterentwicklung der flexiblen Plankostenrechnung und Deckungsbeitragsrechnung. Hierbei sind Betriebs- und Absatzmodelle, die auf Modellen wie dem Hochofenprozessmodell basieren, entscheidend. Sie helfen, Einflussgrö{\ss}en auf den Periodenerfolg zu erfassen und berücksichtigen dabei nur die für die Planung, Überwachung und Produktkalkulation relevanten Beziehungen. Absatzmodelle, die auf Analysen vergangener Prozesse und Prognosen zukünftiger Absatzaktivitäten basieren, sind ebenso bedeutsam. Sie unterstützen marktsegmentspezifische Modellbildung und Prognosen zukünftiger Absatzentwicklungen. Des Weiteren ermöglichen sie fundierte Absatz- und Erlösplanungen, vorausgesetzt, man kennt die bedeutendsten Erlöseinflussgrö{\ss}en und ihre Wirkungszusammenhänge. \footcite[Vgl.][S. 295f]{Artikel_orginal}

Das primäre wirtschaftliche Ziel von Unternehmen ist die Maximierung des Periodenerfolgs. Die BPKER, die auf Betriebs- und Absatzmodellen basiert, zielt auf die Erreichung der optimalen Produktions- und Absatzergebnisse ab. Durch die Überwachung der Umsetzung dieser Pläne können Erwartungen für zukünftige Periodenkosten und -erlöse ermittelt werden. Während die flexible Plankosten- und Deckungsbeitragsrechnung nur die periodenbezogene Produktions- und Absatzplanung berücksichtigt, beinhaltet die BPKER zusätzlich die Planung und Überwachung der produktionsbezogenen Kosten und Absatzerlöse. Sie ermöglicht eine detailliertere Ermittlung der Ursachen von Plan-Ist-Abweichungen und erfüllt alle notwendigen Dokumentationsanforderungen. \footcite[Vgl.][S. 296f]{Artikel_orginal}

\subsubsection{Aufbau der BPKER}

Die Erstellung einer BPKER beginnt mit der Identifizierung wesentlicher Kosten- und Erlöseinflussgrö{\ss}en, die die Ursachen für Kostenverbräuche oder Faktoreinsätze sind. Die Verfügbarkeit dieser Einflussgrö{\ss}en variiert, wobei einige intern kontrolliert und andere exogen bestimmt werden. Neben primären gibt es auch sekundäre Einflussgrö{\ss}en (Zwischenergebnisse in der betrieblichen Einflussgrö{\ss}enrechnung). Ihre Wirkungsweise auf Faktoreinsätze und Absatzleistungen kann durch statistische und analytische Verfahren bestimmt werden. Aufgrund der begrenzten Gültigkeit der statistisch ermittelten Koeffizienten, sind regelmä{\ss}ige Überprüfungen der Einflussgrö{\ss}enfunktionen nötig. Die Strukturmatrix hilft, Produktions- und Absatzprozesse abzubilden indem sie primäre Einflussgrö{\ss}en und abgeleitete Zielgrö{\ss}en darstellt. Sie verfolgt einen ähnlichen Aufbau, wobei das Absatzmengenprogramm im ersten Schritt bestimmt wird. Auftragsbestände können häufig als Grundlage für monatliche und quartalsbezogene Absatz- und Produktionsplanung dienen. Schlie{\ss}lich wird das Absatzprogramm in Bezug auf Vertriebswege, zusätzliche Dienstleistungen und erlösbestimmende Komponenten spezifiziert. \footcite[Vgl.][S. 299ff]{Artikel_orginal}

Betriebs- und Absatzmodelle, die als Input-Output-Modelle fungieren, stellen die Beziehungen zwischen Produktmengen und deren Einflussgrö{\ss}en dar. Mittels multiplikativer Verknüpfung mit Preisvektoren werden differenzierte Planerlöse und Plankosten je Periode generiert. Der Planperiodenerfolg entspricht der Differenz zwischen Gesamtperiodenerlösen und -kosten. Bei aufeinanderfolgenden Periodenmodellen werden Bestandszuführungen und Bestandsentnahmen von Halb- und Fertigfabrikaten berücksichtigt. Es besteht die Möglichkeit, sowohl Alternativkalkulationen durchzuführen als auch Optimierungsrechnungen mittels linearer Programmierungsansätze anzuwenden. \footcite[Vgl.][S. 302]{Artikel_orginal}

\subsubsection{Einsatz der BPKER im Unternehmenscontrolling bei Sorten- und Serienfertigung}

Betriebs- und Absatzmodelle reflektieren Mengenbeziehungen zwischen Produktionsprogramm und wesentlichen Einflussgrö{\ss}en. Änderungen im Produktionsprogramm erfordern Anpassungen wie Verfahrenswechsel oder Varianz von Fertigungsgrö{\ss}en. Dabei sind die Erfolgswirkungen verschiedener Ma{\ss}nahmen entscheidend. Die Mengenberechnung ermittelt Erzeugniseinsatzmengen und Planzeiten, und die Bewertungsrechnung berechnet Periodenkosten durch Anwendung geeigneter Preiskategorien auf Kostengüterbedarfsmengen. Die zunehmende Differenz zwischen Jahres-Plankosten und aktuellem Preisniveau erfordert kurz- und mittelfristige Anpassungen und Optimierungsansätze. Durch Simulationen können die Auswirkungen von Rabatten, Boni und Vertriebswegvariationen auf den Periodenerlös beurteilt werden. \footcite[Vgl.][S. 303ff]{Artikel_orginal}

Die Abweichungsanalyse ist ein Schlüsselwerkzeug für die kurzfristige Erfolgsüberwachung und kann mit unterschiedlicher Granularität durchgeführt werden. Sie hilft dabei, Ursachen und Verantwortlichkeiten für Abweichungen aufzudecken und gliedert Unterschiede zwischen Plan- und Isterfolg nach Erlös- und Kostenabweichungen. Änderungen in den Kostengüterabnahmen aufgrund von veränderten Umständen erfordern Anpassungen. Zudem sind Abweichungen von Sollgrö{\ss}en zu berücksichtigen, insbesondere wenn sie au{\ss}erhalb der statistischen Streubreite liegen. Am Monatsende können anhand der tatsächlichen Produktionsmenge und -bedingungen die Sollkosten berechnet und mit den Istkosten verglichen werden, was wesentliche Abweichungen verdeutlicht. Entscheidungsabhängige Abweichungen ergeben sich durch den Vergleich von Plankosten zu Monatsbeginn und den Sollkosten nach Monatsende. \footcite[Vgl.][S. 305ff]{Artikel_orginal}

Die Erzeugnis- und Auftragskosten und -erlöse sind wesentliche Bestandteile der Kostenträgerrechnung, die zur Preisbildung, Marktpreisbewertung, Lagerbewertung und Erfolgsmessung herangezogen werden. Die BPKER ermöglicht eine flexible Kostenabgrenzung der Kostenträgerrechnung, vorausgesetzt, man setzt für alle Freiheitsgrade einwertige Vorgaben ein. Alternativkalkulationen können in Betracht gezogen werden und Preisänderungen auf die Produktionskosten werden durch die trennende Mengen- und Bewertungsrechnung deutlich. In gleicher Weise erlauben Absatzmodelle Brutto- und Nettoumsatzberechnungen pro Verkaufseinheit. Durch Kombination von Erlös- und Kostenkalkulation wird eine Erfolgskalkulation pro Produkt oder Auftrag ermöglicht, wobei die periodenbezogene Produktions- und Absatzplanung keine Stückerfolgsberechnung erfordert. \footcite[Vgl.][S. 307f]{Artikel_orginal}

\subsubsection{Konzeptionelle Unterschiede zwischen BPKER und flexibler Plankostenrechnung}

Die flexible Plankosten- und Deckungsbeitragsrechnung, die auf der Proportionalität zwischen variablen Kosten und Produktmengenvariationen je Periode basiert, eignet sich vor allem für die Kalkulation, Planung und Kontrolle von Produktions- und Absatzprozessen. Alle anderen Kosten- und Erlöseinflussfaktoren werden im Vorfeld festgelegt. Die BPKER hingegen ermöglicht durch die Erfassung aller wesentlichen Kosten- und Erlösabhängigkeiten die Berechnung des Periodenerfolgs von Planungsvarianten und die detaillierte Untersuchung von Abweichungen, während die Mengen- und Bewertungsrechnung strikt getrennt und laufend mit aktuellen Preisen aktualisiert wird. Sie bietet den Vorteil, Abweichungen nach technischen Einzelfaktoren unter Verantwortlichkeit zu gliedern, während die flexible Plankostenrechnung eine globale Ursachenanalyse ermöglicht. \footcite[Vgl.][S. 308ff]{Artikel_orginal}

\subsubsection{Ergänzung der periodenbezogenen BPKR durch eine Online-Kennziffernrechnung}

In Zeiten steigender Automatisierung, hoher Anlagenintensität und sich ändernder Lohnformen nimmt auch der Anteil der Vorlaufkosten zu. Daher ist eine fein abgestimmte Planung der Kosten essentiell. Die BPKER begegnet diesem Bedarf, indem sie die Kosten- und Erlösabhängigkeiten eines Unternehmens umfassend erfasst und die Berechnung des Periodenerfolgs von Planungsvarianten ermöglicht. Die operative, zeitaktuelle Kennzahlenrechnung ermöglicht eine laufende Überwachung des Produktionsprozesses. Eine Online-Erfassung von Kosten, Erlösen und Mengen- undDasZeitgrö{\ss}en macht Abweichungen von Vorgabewerten sichtbar und ermöglicht rasche Anpassungen. k mit der üblichen Monatsrechnung verbundenes Online-Informationssystem unterstützt bei der Durchführung von Plan-/Soll- und Soll-/Istvergleichen. \footcite[Vgl.][S. 310ff]{Artikel_orginal}

% \begin{figure}[H]
%  \centering
%  \includegraphics[height=5.82cm]{Bilder/REST_Rest.png}
%  \caption[REST Designprinzipien]{REST Designprinzipien. Abgerufen von \cite{fielding_architectural_2000} am 05.07.2023.}
%  \label{fig:iso_norm}
% \end{figure}
