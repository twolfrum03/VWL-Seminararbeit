\chapter{Schlussbetrachtungen}

\section{Bewertung des Ansatzes}



\section{Zusammenfassung}

% Einleitung


% Vorstellung und Einführung in das Thema und den Autor
% Erläuterung der Relevanz und Zielsetzung der Arbeit


% Theoretischer Hintergrund


% Erläuterung der Grundlagen der Betriebsplankosten- und Betriebsplanerfolgsrechnung
% Detaillierte Diskussion der Hauptkonzepte und Prinzipien, die Gert Laßmann in seinem Werk vorstellt, einschließlich Betriebs- und Absatzmodelle, Periodenerfolg, Periodenkosten, Einflußgrößenfunktionen etc.
% Unterschiede zwischen der Betriebsplankosten- und -erfolgsrechnung und der flexiblen Plankosten- und Deckungsbeitragsrechnung


% Anwendung des Theoretischen Konzepts


% Vorstellung eines fiktiven Szenarios/ Unternehmens
% Anwendung der Konzepte von Gert Laßmann auf dieses Szenario, um zu zeigen, wie sie in der Praxis funktionieren würden


% Analyse und Diskussion


% Bewertung der Stärken und Schwächen der von Gert Laßmann vorgestellten Konzepte basierend auf ihrer Anwendung im Szenario
% Diskussion über die Vorzüge der Betriebsplankosten- und Betriebsplanerfolgsrechnung gegenüber anderen Methoden
% Betrachtung möglicher Herausforderungen oder Schwierigkeiten bei der Umsetzung


% Fazit und Ausblick


% Zusammenfassung der wichtigsten Erkenntnisse der Arbeit
% Überlegungen zur weiteren Forschung oder zur praktischen Umsetzung der untersuchten Konzepte


% Literaturverzeichnis

% Diese Gliederung bietet einen umfassenden Überblick über das Thema, ermöglicht eine klare und präzise Darstellung der Theorie, ihre Anwendung in einem praktischen Kontext und eine abschließende Bewertung der Konzepte.



% Betriebsplankostenrechnung nach Gert Laßmann:

% Gert Laßmanns Betriebsplankostenrechnung beinhaltet das Identifizieren der variablen Kosten eines Unternehmens, das heißt, Kosten, die mit der Produktionsmenge variieren. Bei ProdTech GmbH wären dies zum Beispiel die Materialkosten, welche mit der Anzahl der produzierten Einheiten steigen.
% Implementierung am Beispiel der ProdTech GmbH:
% Zuerst würden wir alle Kosten ermitteln, die direkt mit der Produktion in Zusammenhang stehen. Dazu gehören zum Beispiel Materialkosten, Arbeitskosten und Energiekosten. Diese Kosten sind in der Regel variabel, das heißt, sie steigen mit der Menge der produzierten Einheiten.
% Folgendermaßen könnte eine Kostenanalyse für ein einzelnes Produkt, beispielsweise Produkt A, aussehen:

% Materialkosten pro Einheit: 50€
% Arbeitskosten pro Einheit: 25€
% Energiekosten pro Einheit: 15€
% Sonstige Kosten pro Einheit: 10€


% Identifizierung der Einflussgrößen:

% Im nächsten Schritt würden wir die "Einflussgrößen" identifizieren, die Laßmann in seinem Modell beschreibt. Diese können als Ursachen von Kosten angesehen werden. In unserem Fall könnten das zum Beispiel die Art des Rohmaterials, die Produktionsmenge oder die Arbeitszeit sein.
% Zum Beispiel könnte ein bestimmtes Rohmaterial teurer sein, aber die Produktionszeit verkürzen und damit Arbeitskosten sparen. In diesem Schritt würden wir diese Einflussgrößen und ihre Auswirkungen auf die Kosten analysieren.

% Betriebsplanerfolgsrechnung nach Gert Laßmann:

% Das zweite Konzept, das Laßmann vorstellt, ist die Betriebsplanerfolgsrechnung. Bei dieser Methode geht es darum, die voraussichtlichen Erlöse in Bezug auf unterschiedliche Produktionsmengen zu berechnen.
% Angewandt auf die ProdTech GmbH könnte das bedeuten, den geplanten Verkaufspreis (beispielsweise 200€ pro Einheit für Produkt A) zu nehmen und diesen mit der geplanten Verkaufsmenge (zum Beispiel 500 Einheiten) zu multiplizieren. Dadurch erhalten wir den voraussichtlichen Erlös aus dem Verkauf von Produkt A.
% Schließlich würden wir die erwarteten Kosten von den erwarteten Erlösen abziehen, um den voraussichtlichen Gewinn zu berechnen. In diesem Schritt könnten wir dann die Rentabilität der verschiedenen Produktlinien vergleichen und Empfehlungen für das Unternehmen aussprechen.