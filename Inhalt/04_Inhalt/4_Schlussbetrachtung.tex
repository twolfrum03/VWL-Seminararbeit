\chapter{Schlussbetrachtungen}

\section{Bewertung des Ansatzes}

Die BPKER hat sich als ein wichtiges Werkzeug für die Planung und Überwachung von Produktion und Absatz etabliert. Die Methode berücksichtigt alle relevanten Einflussfaktoren auf den Periodenerfolg, was sie zu einem detaillierten Modell für präzise Prognosen und Kontrolle macht.

Ein vorteilhaftes Merkmal dieses Modells ist die Möglichkeit der taggenauen Planung. Dies kommt speziell Branchen mit sich schnell ändernden Produktions- und Absatzbedingungen zugute, wie zum Beispiel der Montanindustrie (Bergbau-, Stahlindustrie), in der sich externe Bedingungen schnell und erheblich ändern können. La{\ss}manns Methode hält dank der angelegten Flexibilität mit den Veränderungen Schritt und ermöglicht dadurch bessere Vorhersagen und Anpassungen.

Eine weitere Stärke des BPKER liegt in ihrer Integration von Kostenrechnung und wesentlichen Planungssystemen. Dieser hollistische Ansatz führt zu einer ganzheitlichen Darstellung der variablen und festen Kosten sowie der Modulebene Kostenanalyse. In anderen Modellen, wie der Grenzplankostenrechnung, werden Kosten- und Erlösrechnung oft getrennt behandelt, wodurch ein Teil der Information verloren gehen kann.

Dennoch gibt es auch einige Nachteile bei der Verwendung der BPKER. Eines der Hauptprobleme ist der hohe Aufwand, der mit der Durchführung verbunden ist. Idealerweise wird das Modell möglichst oft mit allen relevanten Daten aktualisiert, um die genauesten Vorhersagen und Überwachungen für Produktion und Absatz zu ermöglichen. In der Praxis kann dieser Aktualisierungsprozess jedoch aufgrund der erforderlichen Ressourcen und des hohen Detailgrades zeitaufwändig und kostspielig sein.

Ein weiterer Nachteil des Modells ist seine Komplexität. Die Berücksichtigung aller Einflussfaktoren führt zu einem komplexen Modell, das spezielles Fachwissen für seine Umsetzung und Überwachung erfordert. Das hindert kleinere Unternehmen am Zugang. Die Verwendung des Modells ist auch dadurch eingeschränkt, dass es auf die spezifischen Produktions- und Absatzbedingungen der Montanindustrie zugeschnitten ist.

Au{\ss}erdem, obwohl La{\ss}manns Modell bei gro{\ss}en Unternehmen in der Montanindustrie implementiert wurde, wird die Methode insgesamt nicht so breit akzeptiert und angewandt wie andere Modelle (Grenzplankostenrechnung), woran höchstwahrscheinlich auch die höheren Ressourcenanforderungen und die grö{\ss}ere Komplexität des Modells schuld sein könnten.

Zusammenfassend lässt sich sagen, dass die BPKER ein effektives Werkzeug zur detaillierten Planung und Überwachung von Produktion und Absatz ist. Es ermöglicht genaue Prognosen und bietet einen umfassenden Überblick über die Kosten und Erlöse. Allerdings sind die Ressourcenanforderungen und die Komplexität des Modells erhebliche Hindernisse für seine Anwendung, insbesondere in kleineren Unternehmen. Daher muss jedes Unternehmen sorgfältig abwägen, ob die Vorteile die potenziellen Nachteile überwiegen.

\section{Reflexion der Arbeit und Ausblick}

Bei der Untersuchung der BPKER wurden sowohl deren Potenziale als auch Herausforderungen deutlich. Es wurde veranschaulicht, wie dieses Modell durch die Berücksichtigung relevanter Einflussgrößen und die detaillierte Darstellung von Produktions- und Absatzbedingungen eine umfassende Struktur für fundierte Unternehmensentscheidungen bietet. Gleichzeitig zeigte sich die Herausforderung der hohen Komplexität und der nicht zu vernachlässigbaren Ressourcen, die für die Implementierung dieses Ansatzes erforderlich sind.

Die Anwendung des Modells auf die HobelKunstwerkstatt GmbH beleuchtete die potenzielle Realität einer solchen Anwendung. Es wurden wichtige Erkenntnisse darüber gewonnen, wie das Modell in der Praxis funktioniert und welche spezifischen Anforderungen und Hürden bei der Anwendung auftreten können. Zudem wurde deutlich, dass eine individuelle Anpassung an die Besonderheiten eines Unternehmens oder einer Branche wesentlich für den erfolgreichen Einsatz des Modells ist.

In Bezug auf zukünftige Perspektiven und die Weiterentwicklung der BPKER besteht sicherlich ein erheblicher Forschungsbedarf. Jedoch bestehen im Hinblick auf die bereits weitreichende Verbreitung anderer Modelle und die fast schon in Vergessenheit geratene BPKER Zweifel daran, dass das Modell tatsächlich weiterentwickelt wird. Es wäre um es praktikabel einsetzbar zu machen wichtig, Methoden zu entwickeln, um die Effizienz des Modells zu erhöhen. In diesem Zusammenhang könnten Ansätze zur Integration von Technologien wie künstlicher Intelligenz und Data-Mining-Verfahren eine interessante Perspektive bieten. Zudem müsste das Modell für eine breite Pallette an Industrien angepasst werden, um ein Fundament für eine breite Akzeptanz zu schaffen. 

Zusammenfassend bietet die BPKER eine detaillierte und umfassende Methode zur Planung und Kontrolle von Betriebsprozessen. Obwohl sie Herausforderungen mit sich bringt, bietet sie zugleich umfangreiche Möglichkeiten für die Betriebswirtschaft. Es bleibt spannend, wie sich die Forschung und Anwendung dieses Modells in der Zukunft weiterentwickeln werden.